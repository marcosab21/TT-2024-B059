\chapter{Resumen}
El presente proyecto documenta el diseño y desarrollo de un simulador de laboratorio de química inorgánica en realidad virtual, enfocado en la enseñanza interactiva y visualización inmersiva de procesos químicos. El prototipo utiliza tecnologías avanzadas como el seguimiento de manos (hand tracking) para facilitar la manipulación directa de objetos virtuales, una tabla periódica interactiva para seleccionar elementos químicos y sistemas de máquinas de estados finitos (FSM) para gestionar el flujo de los experimentos.

El simulador incluye un tutorial y cuatro experimentos estructurados en fases, que guían al usuario en tareas como el balanceo de ecuaciones químicas, la creación de compuestos y la ejecución de reacciones simuladas. Estas fases integran efectos visuales en tiempo real (VFX) para representar fenómenos físicos, acompañados de retroalimentación auditiva para reforzar la inmersión.

El desarrollo del proyecto implicó desafíos técnicos multidisciplinarios, abarcando áreas como modelado 3D, desarrollo de interfaces, implementación de sistemas interactivos y validación de precisión química. A pesar de estos retos, se logró cumplir con los objetivos establecidos, entregando un prototipo funcional validado por expertos en química y probado por múltiples usuarios. Este trabajo establece un punto de partida para futuras mejoras, incluyendo la expansión del catálogo de experimentos, la incorporación de herramientas de evaluación del aprendizaje, la compatibilidad con diversos dispositivos de realidad virtual y la implementación de colaboración en línea.