\chapter{Agradecimientos}
Este proyecto no habría sido posible sin el apoyo, guía y motivación de muchas personas e instituciones a lo largo de este camino. A todas ellas, quiero expresar mi más profundo agradecimiento.

En primer lugar, mi gratitud infinita a mis directores, el \textbf{Maestro Saúl de la O Torres} y el \textbf{Doctor Gabriel Sepúlveda Cervantes}, por su invaluable guía, dedicación y apoyo constante. Su experiencia y compromiso no solo orientaron este trabajo, sino que también dejaron una huella imborrable en mi formación profesional.

Al \textbf{Instituto Politécnico Nacional}, por abrirme sus puertas y brindarme durante los últimos seis años un espacio para crecer, aprender y soñar. Gracias por proporcionarme las herramientas, conocimientos y valores necesarios para poner la técnica al servicio de la patria y contribuir a un mundo mejor.

A mi familia, que ha estado conmigo en las buenas, en las malas y en las peores, mi más sincero agradecimiento. A mi madre, \textbf{María Luisa Aguayo Torres}, mi mayor inspiración y guía. Su amor incondicional y esfuerzo incansable han sido la base de cada logro que he alcanzado. A mi padre, \textbf{Luis Carlos García Alaniz}, por su apoyo constante y por estar presente cuando más lo he necesitado. A mis hermanos, \textbf{Carlos Pavel Cuauhtémoc Tupac Amaru} y \textbf{María Sachenka América Libertad}, por ser una fuente constante de motivación, cariño y respaldo en cada etapa de este camino.

A mis compañeros del \textbf{Laboratorio de Realidad Virtual del CIDETEC}, por hacer cada tarde más amena con sus consejos, historias y camaradería. Gracias, \textbf{Gus, Ángel, Germán, Pepe, Lalo, Mike, Gabo, Steven, Wonka y Fer}, por convertir cada jornada de trabajo en una experiencia inolvidable, llena de aprendizajes y risas.

A la \textbf{Escuela Secundaria Diurna No. 4 "Moisés Sáenz"}, por abrirme sus puertas y permitirme llevar este proyecto a la práctica, facilitando un espacio para su desarrollo y pruebas.

Al \textbf{profesor Ernesto Ángel Uribe Pérez}, por su generosidad al compartir su vasto conocimiento en química, sus valiosas ideas y su apoyo constante. Su experiencia fue un aporte esencial para el éxito de este proyecto.

A la alumna \textbf{Castro Gordillo Miranda} del grupo \textbf{3° A}, por su dedicación y tiempo en la realización de las pruebas que contribuyeron significativamente al desarrollo del simulador.

Finalmente, a todos los profesores que he tenido el honor de tener a lo largo de mi formación académica. Su dedicación y pasión por la enseñanza dejaron en mí una semilla de curiosidad y amor por el conocimiento que sigue creciendo cada día.

Este logro es el resultado de un esfuerzo colectivo y una suma de voluntades.
\newpage