\thispagestyle{empty}.
    % Cabecera: imágenes y texto superior alineados horizontalmente
    \noindent
    \begin{adjustbox}{max width=\linewidth}
        \begin{minipage}[c][5cm][c]{0.15\textwidth} % Imagen izquierda
            \centering
            \includegraphics[height=3.5cm,keepaspectratio]{img/logo-ipn.png}
        \end{minipage}
        \begin{minipage}[c][5cm][c]{0.75\textwidth} % Texto central
            \centering
            \linespread{2}\selectfont
            {\fontsize{20}{1}\selectfont\textbf{\textsc{Instituto Politécnico Nacional\\}}} 
            {\fontsize{16}{15}\selectfont\textbf{\textsc{ESCUELA SUPERIOR DE CÓMPUTO\\}}} 
            {\fontsize{16}{15}\selectfont\textbf{\textsc{SUBDIRECCIÓN ACADÉMICA\\}}}
        \end{minipage}
        \begin{minipage}[c][5cm][c]{0.15\textwidth} % Imagen derecha
            \centering
            \includegraphics[height=2.25cm,keepaspectratio]{img/logo-school.png}
        \end{minipage}
    \end{adjustbox}

    \vspace{1cm} % Espaciado entre la cabecera y el contenido principal

    % Contenido principal fuera de los minipages
    \begin{multicols}{2}
        \begin{flushleft}
            {\fontsize{14}{14}\selectfont{No. De TT: 2024-B059}}
        \end{flushleft}
    
        \begin{flushright}
            {\fontsize{14}{14}\selectfont{19 de diciembre de 2024}}
        \end{flushright}
    \end{multicols}
    \centering
    {\fontsize{14}{14}\selectfont\textsc{Documento Técnico\\}}
    \vspace{20pt}
    {\fontsize{16}{16}\selectfont\textbf{``Prototipo de Simulador de Laboratorio de Química Inorgánica en Realidad Virtual''\\}}

    \vspace{40pt}
    {\fontsize{14}{14}\selectfont\textsc{Presenta:\\}}
    \vspace{10pt}
    {\fontsize{14}{14}\selectfont\textbf{\doclink{García Aguayo Marcos Martí Sandino Mictlantecuhtli}{marcos.ab21@hotmail.com}}}

    \vspace{40pt}
    {\fontsize{14}{14}\selectfont\textsc{Directores:\\}}
    \begin{multicols}{2}
        {\fontsize{14}{14}\selectfont\textbf{M. en E. Saul De La O Torres}\\}
        {\fontsize{14}{14}\selectfont\textbf{Dr. Gabriel Sepúlveda Cervantes}}
    \end{multicols}

    \vspace{30pt}
    {\fontsize{14}{14}\selectfont\textsc{Resumen:\\}}
    \vspace{10pt}
    \justify
    {\fontsize{14}{14}\selectfont
    Este trabajo presenta el desarrollo de un prototipo de simulador de laboratorio de química inorgánica en realidad virtual, orientado a la enseñanza inmersiva mediante tecnologías avanzadas como seguimiento de manos, sistemas FSM y una tabla periódica interactiva. El sistema permite realizar experimentos estructurados que abarcan balanceo de ecuaciones, creación de compuestos y simulación de reacciones químicas, integrando efectos visuales y auditivos en tiempo real. Validado por expertos.}

    {\fontsize{14}{14}\selectfont Palabras clave: Química Inorgánica, Realidad Virtual, Simulador, Software Interactivo.\\}
    
    \vspace{20pt}
    \raggedright
    
