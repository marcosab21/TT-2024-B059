\chapter{Trabajo a Futuro}\label{ch:Trabajo_Futuro}
El prototipo desarrollado presenta un punto de partida sólido para la enseñanza inmersiva de la química inorgánica. Sin embargo, se han identificado áreas clave que podrían ser abordadas en futuras iteraciones para ampliar su funcionalidad, mejorar su impacto educativo y adaptarse a un público más amplio.
\begin{enumerate}[I.]
    \item \textbf{Ampliación del Catálogo de Experimentos}\\
    Actualmente, el simulador incluye cuatro experimentos básicos. Un paso importante a futuro sería la incorporación de nuevos experimentos que abarquen conceptos avanzados de química, incluyendo experimentos que involucren química orgánica o reacciones más complejas. Esto permitiría que el simulador sea útil no solo en el nivel secundaria, sino también en niveles educativos más avanzados, como bachillerato o universidad.
    \item \textbf{Evaluación del Aprendizaje}\\
    Integrar un sistema de evaluación del aprendizaje proporcionaría una retroalimentación inmediata sobre el desempeño de los usuarios. Esto podría incluir cuestionarios al final de cada experimento, análisis de las decisiones tomadas durante las prácticas y la generación de reportes de progreso. Un sistema de este tipo no solo reforzaría el aprendizaje, sino que también permitiría a los docentes medir el impacto del simulador en el desarrollo de habilidades químicas en los estudiantes.
    \item \textbf{Compatibilidad con Diversos Equipos de RV}\\
    Para ampliar el alcance del simulador, sería importante trabajar en la compatibilidad con una mayor variedad de dispositivos de realidad virtual, como Meta Quest, HTC Vive, o dispositivos más económicos que utilizan smartphones como visores. Esta flexibilidad tecnológica permitiría que más instituciones educativas, independientemente de sus recursos, puedan implementar el simulador en sus aulas.
    \item \textbf{Incorporación de Colaboración en Línea}\\
    El desarrollo de un modo multijugador o colaborativo en línea sería un avance significativo. Esto permitiría que varios usuarios trabajen juntos en experimentos en tiempo real, simulando prácticas grupales que son comunes en laboratorios físicos. Además, este enfoque fomentaría el aprendizaje cooperativo y permitiría la participación remota, haciendo el simulador accesible para estudiantes en ubicaciones geográficas distintas.
\end{enumerate}
Estas propuestas de trabajo a futuro buscan consolidar el simulador como una herramienta educativa avanzada y versátil, capaz de adaptarse a diversas necesidades y contextos. Su implementación podría transformar el prototipo en una plataforma integral para el aprendizaje de la química, 