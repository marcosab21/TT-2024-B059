\chapter{Marco Teórico}\label{ch:Marco_Teórico}

\section{Química}
La química es la ciencia que estudia la materia, su composición, estructura, propiedades y las transformaciones que experimenta. En el contexto de este proyecto, nos centraremos en los aspectos de la química inorgánica relevantes para los experimentos que se simularán en el entorno virtual. 

\subsection{Química Inorgánica}

Constituye una de las ramas principales de la química, enfocada en el estudio de los elementos y compuestos que no contienen carbono, excluyendo así los compuestos orgánicos. Su ámbito de estudio abarca desde la comprensión de las propiedades básicas de los elementos individuales hasta la síntesis y caracterización de compuestos inorgánicos complejos.

\subsection{Reacciones Químicas}
Las reacciones químicas representan procesos en los cuales sustancias reactantes se transforman en otras sustancias, denominadas productos. Estas transformaciones se expresan mediante ecuaciones químicas, donde las fórmulas de los reactivos se encuentran en el primer miembro y las de los productos en el segundo. Los coeficientes en estas ecuaciones indican las proporciones molares de las sustancias participantes \cite{alfa_nauta}.

\begin{itemize}
    \item \textit{\textbf{Reacciones de Oxidación-Reducción (Redox): }}
    Una reacción de oxidación-reducción (redox) es un tipo de reacción química que implica la transferencia de electrones entre dos especies. La especie que pierde electrones se oxida, mientras que la especie que gana electrones se reduce. Estas reacciones son fundamentales en muchos procesos químicos y biológicos, y su comprensión es crucial para el estudio de la química. 

    En una reacción redox, el número de electrones perdidos por la especie que se oxida debe ser igual al número de electrones ganados por la especie que se reduce. Esto se debe a la ley de conservación de la masa, que establece que la materia no se crea ni se destruye en una reacción química, solo se transforma. 
    
    Para identificar una reacción redox, se pueden utilizar los números de oxidación, que indican el estado de oxidación de un átomo en un compuesto. El número de oxidación de un átomo puede ser positivo, negativo o cero, y representa la carga hipotética que tendría el átomo si todos sus enlaces fueran iónicos. 
    \item \textit{\textbf{Reacciones de Combustión: }}
    Las reacciones de combustión son un tipo particular de reacción redox en las que una sustancia combustible reacciona con un oxidante (generalmente oxígeno) para producir óxido del combustible y liberar energía en forma de luz y calor. 
    
    \textit{\textbf{Energía de activación y punto de ignición }}\\
    Para que una reacción de combustión se inicie, se requiere un aporte inicial de energía, llamada energía de activación. Esta energía puede provenir de una llama, chispa o fuente de calor. Una vez iniciada la reacción, la energía liberada en forma de calor es suficiente para mantener la combustión hasta que se agote el combustible. 
    El punto de ignición es la temperatura mínima a la que una sustancia combustible debe ser calentada para iniciar la combustión en presencia de oxígeno. Cada combustible tiene un punto de ignición específico.  
    \begin{itemize}
        \item \textit{\textbf{Reacciones exotérmicas:}} Liberan energía al entorno en forma de calor.
        \item \textit{\textbf{Reacciones endotérmicas:}} Absorben energía del entorno.
    \end{itemize}
    \item \textit{\textbf{Reacciones de Síntesis: }}
    Una reacción de síntesis es aquella en la que dos o más sustancias (reactivos) se combinan para formar un único producto. En general, estas reacciones siguen el patrón: A + B → AB
\end{itemize}

\subsection{Estequiometría}
La estequiometría es una herramienta esencial en química que permite analizar las relaciones cuantitativas entre reactivos y productos en una reacción química. Se basa en el concepto de mol, que es la unidad fundamental de cantidad de sustancia en el Sistema Internacional de Unidades (SI). Un mol de cualquier sustancia contiene el mismo número de entidades elementales (átomos, moléculas, iones, etc.), conocido como número de Avogadro ($6.022 * 10^{23}$)\cite{Estequiometria}. 
    
La estequiometría se utiliza para: 
    
\textbf{Balancear ecuaciones químicas:} Ajustar los coeficientes de una ecuación química para que el número de átomos de cada elemento sea el mismo en ambos lados de la ecuación, cumpliendo así con la ley de conservación de la masa. 
    
\textbf{Calcular relaciones molares:} Determinar la cantidad de moles de una sustancia que reaccionan con una cantidad dada de otra sustancia, o la cantidad de moles de producto que se forman a partir de una cantidad dada de reactivo. 
    
\textbf{Calcular relaciones de masa:} Convertir entre masa y moles de una sustancia utilizando la masa molar, que es la masa de un mol de esa sustancia. 
    
\textbf{Calcular el rendimiento de una reacción:} Determinar la cantidad de producto que se obtiene en una reacción química en comparación con la cantidad teórica que se debería obtener según la estequiometría. 
    
La estequiometría es una herramienta fundamental en el laboratorio, ya que permite a los químicos diseñar experimentos, predecir los resultados y analizar los datos obtenidos. 

\subsection{Cinética Química}
La cinética química es el estudio de la velocidad de las reacciones químicas y los factores que la afectan. La velocidad de una reacción se define como el cambio en la concentración de un reactivo o producto por unidad de tiempo\cite{Cinetica_Quimica_Basica}. 

Los factores que afectan la velocidad de una reacción química incluyen: 

\begin{itemize}
    \item \textbf{Concentración de los reactivos:} En general, la velocidad de una reacción aumenta a medida que aumenta la concentración de los reactivos. Esto se debe a que una mayor concentración de reactivos significa que hay más moléculas disponibles para colisionar y reaccionar. 

    \item \textbf{Temperatura:} La velocidad de una reacción generalmente aumenta a medida que aumenta la temperatura. Esto se debe a que las moléculas se mueven más rápido a temperaturas más altas, lo que aumenta la frecuencia y la energía de las colisiones. 

    \item \textbf{Catalizadores:} Un catalizador es una sustancia que aumenta la velocidad de una reacción química sin consumirse en ella. Los catalizadores funcionan al proporcionar un camino de reacción alternativo con una menor energía de activación. 
\end{itemize}

\subsection{Ley de Rapidez}
La ley de rapidez, también conocida como ecuación de velocidad, describe la relación matemática entre la velocidad de una reacción química y la concentración de los reactivos. Esta ley se expresa de la siguiente manera: 

Velocidad de reacción = $k[A]^m[B]^n...$ 
 

Donde: 

k: Es la constante de velocidad de la reacción, que depende de la temperatura y la naturaleza de los reactivos. 

[A], [B], ...: Son las concentraciones molares de los reactivos A, B, etc. 

m, n, ...: Son los órdenes de reacción respecto a cada reactivo, que indican cómo la concentración de cada reactivo afecta la velocidad de la reacción. Estos órdenes de reacción se determinan experimentalmente. 

La ley de rapidez es una herramienta fundamental en la cinética química, ya que permite predecir cómo cambiará la velocidad de una reacción al variar las concentraciones de los reactivos.

\subsection{Leyes ponderales}
También conocidas como leyes de las combinaciones químicas, son principios fundamentales que describen las relaciones cuantitativas entre los reactantes y productos en las reacciones químicas. Estas leyes fueron establecidas por John Dalton a principios del siglo XIX y sentaron las bases para la estequiometría, la rama de la química que se ocupa de las relaciones cuantitativas en las reacciones químicas\cite{Fernando2021}.
\newpage
\textbf{\textit{Ley de la Conservación de la Masa}}\\
En toda reacción química ordinaria, la masa total de los reactantes es igual a la masa total de los productos. Esto significa que la materia no se crea ni se destruye durante la reacción, solo se transforma de una forma a otra.
\\\\
\textbf{\textit{Ley de las Proporciones Definidas}}\\
Los elementos químicos se combinan para formar compuestos en proporciones fijas y definidas. Esto significa que la composición de un compuesto puro siempre es la misma, independientemente de cómo se haya obtenido.
\\\\
\textbf{\textit{Ley de las Proporciones Múltiples}}\\
Cuando dos elementos forman más de un compuesto, las masas de uno de ellos que se combinan con una masa fija del otro elemento guardan una relación sencilla de números enteros.

\section{Simulador}
Los simuladores son herramientas fundamentales en campos diversos, donde ofrecen una recreación detallada y precisa de sistemas complejos. Al recrear sensaciones físicas y comportamientos específicos, los simuladores permiten a los usuarios experimentar situaciones de manera segura y controlada. Desde entrenamientos de pilotos hasta pruebas de diseño de productos, la simulación proporciona un entorno virtual que facilita la comprensión del comportamiento de sistemas complejos y la evaluación de estrategias de operación. Además, los simuladores son una parte crucial de la innovación y el desarrollo, al permitir experimentar con diferentes escenarios sin arriesgar recursos o vidas humanas.

\subsection{Elementos de una simulación \cite{Simulación}}
\begin{enumerate} [I. ]
    \item Definición del Sistema: 
    Antes de adentrarnos en la simulación del sistema, es crucial realizar un análisis preliminar exhaustivo. Este análisis nos permite entender la interacción del sistema con otros, identificar restricciones, determinar las variables involucradas y sus relaciones, así como establecer las medidas de efectividad y los resultados esperados.
    \item Formulación del Modelo: 
    Una vez clarificados los objetivos del estudio, pasamos a la etapa de construcción del modelo. Aquí, se definen meticulosamente todas las variables que conforman el sistema, sus interrelaciones y se elaboran diagramas de flujo que retraten de manera integral el modelo en cuestión. Este proceso es fundamental para garantizar resultados precisos y útiles.
    \item Colección de Datos: 
    La disponibilidad y accesibilidad de ciertos datos, así como la complejidad para obtener otros, pueden incidir en la formulación y desarrollo del modelo. Por ello, es crucial definir con precisión los datos necesarios para lograr los resultados deseados. Habitualmente, esta información se obtiene de registros contables, órdenes de trabajo, órdenes de compra, opiniones de expertos y, en última instancia, a través de experimentación si es imprescindible. Es fundamental esta etapa para asegurar la solidez y fiabilidad del modelo.
    \item Implementación del Modelo en la Computadora:
    Una vez que el modelo está definido, el siguiente paso implica la elección entre el desarrollo en un lenguaje de programación específico o la utilización de un paquete de software adecuado para su procesamiento en la computadora, con el fin de obtener los resultados deseados.
    \item Validación: 
    Durante esta etapa, se pueden identificar posibles deficiencias tanto en la formulación del modelo como en los datos utilizados para alimentarlo. Las formas más comunes de validar un modelo son las siguientes:
    \begin{itemize}
        \item Evaluación por expertos de los resultados de la simulación.
        \item Comparación de la precisión en la predicción de datos históricos.
        \item Evaluación de la precisión en la predicción de eventos futuros.
        \item Comprobación de la falla del modelo de simulación al utilizar datos que causen fallas en el sistema real.
        \item Aceptación y confianza en el modelo por parte de las personas que utilizarán los resultados generados por el experimento de simulación.
    \end{itemize}
    \item Experimentación: 
    Una vez validado el modelo, se procede a la experimentación. Esta fase implica la generación de los datos necesarios y la realización de análisis de sensibilidad de los indicadores pertinentes. 
    \item Interpretación: 
    Una vez completada la experimentación y obtenidos los resultados de la simulación, se procede a su interpretación. En esta fase, se analizan detalladamente los datos generados y se extraen conclusiones relevantes. Basándose en esta interpretación, se toma una decisión informada.
    \item Documentación: 
    Para maximizar el uso del modelo de simulación, se requieren dos tipos de documentación: técnica y manual de usuario. La documentación técnica detalla el funcionamiento interno del modelo, mientras que el manual de usuario proporciona instrucciones prácticas para su uso efectivo. Juntos, estos documentos garantizan que el modelo sea accesible y útil para todos los usuarios, facilitando su aplicación en diversas situaciones.
\end{enumerate}
\subsection{Beneficios}
\begin{itemize}
    \item Mayor seguridad: 
    Permiten a los usuarios practicar en entornos seguros, evitando riesgos y daños.
    \item Menor costo: 
    Su desarrollo y uso puede ser más económico que realizar experimentos en el mundo real. 
    \item Mayor control: Permiten controlar las condiciones de la simulación, lo que facilita la investigación y el análisis.
    \item Mayor accesibilidad: Pueden ser utilizados por personas que no podrían acceder a experiencias en el mundo real.
\end{itemize}
\subsection{Limitaciones}
\begin{itemize}
    \item Complejidad: 
    La creación de simuladores realistas y precisos puede ser un proceso complejo y costoso.
    \item Fidelidad: 
    Siempre tendrán un cierto grado de abstracción, lo que puede limitar su precisión.
    \item Sesgos: 
    Pueden reflejar los sesgos de sus creadores, lo que puede llevar a resultados inexactos o discriminatorios. 
\end{itemize}

\section{Realidad Virtual (RV)}
La Realidad Virtual (RV) es una tecnología que crea entornos digitales simulados e interactivos, diseñados para ser experimentados e interactuados como si fueran reales. Estos entornos se generan mediante computadoras y se presentan a los usuarios a través de dispositivos como cascos de visualización (HMD) y auriculares, sumergiéndolos en mundos virtuales tridimensionales donde pueden manipular objetos virtuales utilizando controladores de movimiento u otros dispositivos de entrada\cite{VR-Book}.
\subsection{Componentes de un Sistema de RV}
Un sistema de RV típico se compone de varios elementos clave que trabajan en conjunto para crear la ilusión de realidad:
\begin{itemize}
    \item \textit{\textbf{Dispositivos de visualización:}} Cascos de realidad virtual (HMD) que presentan imágenes estereoscópicas, engañando al cerebro para que perciba profundidad y distancia, creando la sensación de estar dentro del entorno virtual.
    \item \textit{\textbf{Dispositivos de audio:}} Auriculares o altavoces espaciales que brindan sonido envolvente, mejorando la inmersión al simular la ubicación y dirección de los sonidos en el entorno virtual.
    \item \textit{\textbf{Dispositivos de seguimiento:}} Sensores que rastrean los movimientos del usuario, como la posición de la cabeza y las manos, y los traducen en acciones dentro del entorno virtual, permitiendo una interacción natural y fluida.
    \item \textit{\textbf{Software y contenido:}} El software genera el entorno virtual, incluyendo modelos 3D, texturas, iluminación e interacciones. El contenido puede ser diverso, desde simulaciones científicas y educativas hasta juegos y experiencias artísticas.
\end{itemize}

\subsection{Inmersión e Interacción}
La inmersión es un componente fundamental de la RV y se refiere al grado en que un sistema de RV proyecta estímulos a los sentidos del usuario de manera extensa, envolvente, vívida e interactiva. La inmersión tiene como objetivo crear una sensación de presencia, donde los usuarios sientan que están realmente presentes en el entorno virtual.

La interacción en la RV es la comunicación entre el usuario y la aplicación de RV, mediada por dispositivos de entrada y salida. La fidelidad de interacción se refiere a la correspondencia entre las acciones físicas en el mundo virtual y las acciones físicas equivalentes en el mundo real. Existen interacciones realistas (alta fidelidad), no realistas (baja fidelidad) y mágicas (nivel medio de fidelidad).
\subsection{Aplicaciones de la RV en la Educación}
En el campo de la educación, la RV ofrece un enorme potencial para mejorar el aprendizaje al permitir a los estudiantes experimentar conceptos abstractos y complejos de una manera más tangible e interactiva.

La RV puede aprovechar las fortalezas de la percepción humana, como la percepción espacial, la percepción del tiempo y la percepción del movimiento, para crear un entorno virtual convincente y realista. Además, la incorporación de elementos de gamificación puede aumentar el compromiso y la motivación de los estudiantes, haciendo que el aprendizaje sea más divertido y efectivo.

\section{Laboratorio Virtual}
Constituyen entornos de aprendizaje interactivos que integran tecnología, pedagogía y elementos humanos para facilitar experiencias prácticas en un ambiente virtual. Su objetivo principal es introducir a los usuarios en la experimentación, resolución de problemas, deducción de resultados e interpretación científica a través de simulaciones.

\subsection{Importancia de los Laboratorios Virtuales}
Los laboratorios virtuales se han convertido en un complemento esencial en la formación experimental en ciencias, ingeniería y tecnología, gracias a los avances en tecnologías de la información y la creciente demanda de educación a distancia. Su importancia se ha visto reforzada por la pandemia de COVID-19. A continuación, se presentan las principales ventajas de estos laboratorios\cite{Importancia_de_los_laboratorios_virtuales}:

\begin{itemize}
    \item\textbf{Mejora del Aprendizaje: }Los laboratorios virtuales permiten la visualización de fenómenos complejos de manera más clara y accesible, lo que facilita la comprensión profunda y la retención de conceptos teóricos.
    \item\textbf{Eficiencia logística y económica: }Representan un ahorro considerable en costos de infraestructura y materiales para las instituciones educativas, lo que puede traducirse en una reducción de las tasas de matrícula. Además, eliminan la necesidad de desplazamientos físicos, lo que se traduce en un ahorro de tiempo y recursos para los estudiantes.
    \item\textbf{Potenciación de la experimentación: }La supervisión docente en el uso de laboratorios virtuales maximiza los resultados de aprendizaje, manteniendo el interés y el compromiso de los estudiantes al proporcionarles un entorno seguro para explorar y aprender de los errores.
\end{itemize}

Si bien los laboratorios virtuales ofrecen un amplio abanico de beneficios, es fundamental considerarlos como un complemento valioso, más no un sustituto, de los laboratorios presenciales. Estos últimos siguen siendo esenciales para el desarrollo de habilidades prácticas y manipulativas que los entornos virtuales aún no pueden replicar en su totalidad. La integración estratégica de ambas modalidades puede enriquecer significativamente la experiencia educativa y preparar a los estudiantes para los desafíos del mundo real.
\newpage
\subsection{Limitaciones en el uso de laboratorios virtuales}
\textbf{\textit{Simplicidad del modelo}}\\
La naturaleza virtual de los laboratorios implica una simplificación inevitable de los fenómenos y procesos reales. Esto puede conllevar una pérdida de información crucial y una distorsión de la complejidad inherente a los sistemas estudiados.

\textbf{\textit{Selección y evaluación rigurosa}}\\
La idoneidad de un laboratorio virtual para una experiencia educativa específica no es universal. Se requiere una evaluación crítica por parte del docente para seleccionar la herramienta adecuada, considerando factores como los objetivos de aprendizaje, el nivel de los estudiantes y la correspondencia con los contenidos curriculares.

\textbf{\textit{Competencias docentes y soporte técnico}}\\
 La efectiva implementación de laboratorios virtuales exige competencias docentes en el manejo de las TIC y en la integración de estas herramientas en la pedagogía. En algunos casos, puede requerirse soporte técnico adicional para garantizar el correcto funcionamiento de las simulaciones y la resolución de problemas informáticos.
 
\textbf{\textit{Necesidad de tutoría}}\\
La autonomía del estudiante en el uso de laboratorios virtuales puede ser limitada, requiriendo la tutoría o guía del docente para una adecuada comprensión y aprovechamiento de las simulaciones. Esto implica una inversión de tiempo y recursos por parte del educador.

\textbf{\textit{Atractivo de los productos}}\\La naturaleza digital de los resultados obtenidos en laboratorios virtuales puede restarles atractivo en comparación con los productos tangibles del laboratorio real. Esto puede afectar la motivación y el engagement de los estudiantes.

\textbf{\textit{Limitación en la manipulación}}\\
Las simulaciones virtuales no permiten la manipulación directa de equipos e instrumentos de laboratorio, lo que puede ser una desventaja para el desarrollo de habilidades prácticas y la adquisición de destrezas procedimentales.
\newpage
\section{Modelado y Diseño}
Para crear contenido, efectos y animaciones, es necesario desarrollar un software específico para modelos 3D. Estos programas se diseñan con el propósito de simular objetos del mundo real. Uno de los aspectos más importantes para dar credibilidad a las imágenes generadas por computadora, junto con la iluminación, son las texturas aplicadas a los objetos.

\subsection{Modelado 3D}
El modelado 3D ha revolucionado diversos sectores, introduciendo una nueva era de diseño y visualización. Mediante software especializado, se crean representaciones matemáticas precisas de objetos tridimensionales, denominados modelos 3D. Estos modelos poseen un gran valor en múltiples industrias:
\begin{itemize}
    \item Industria cinematográfica, televisiva y de videojuegos
    \item Arquitectura, ingeniería y construcción
    \item Diseño de productos
    \item Ciencia y medicina
\end{itemize}

En la \autoref{tab:comparativa3D}, se describe una breve comparación de las características de tres Aplicaciones para el modelado 3D, considerando el sistema donde corre, el costo de la licencia (Pesos mexicanos, precio por año), lenguaje de programación y su documentación. 

\begin{table}[H]
  \centering
  \begin{tabular}{|>{\centering\arraybackslash}m{.1\textwidth}|>{\centering\arraybackslash}m{.03\textwidth}|>{\centering\arraybackslash}m{.03\textwidth}|>{\centering\arraybackslash}m{.03\textwidth}|>{\centering\arraybackslash}m{.09\textwidth}|>{\centering\arraybackslash}m{.02\textwidth}|>
  {\centering\arraybackslash}m{.02\textwidth}|>
  {\centering\arraybackslash}m{.02\textwidth}|>
  {\centering\arraybackslash}m{.02\textwidth}|>
  {\centering\arraybackslash}m{.05\textwidth}|>
  {\centering\arraybackslash}m{.06\textwidth}|>
  {\centering\arraybackslash}m{.06\textwidth}|>
  {\centering\arraybackslash}m{.03\textwidth}|}
    \hline
     & \multicolumn{3}{c|}{\raggedright\parbox[0.12\textwidth]{0.12\textwidth}{Sistema Operativo}} 
     & 
     & \multicolumn{5}{c|}{\raggedright\parbox[0.15\textwidth]{0.15\textwidth}{Lenguajes de Programación}} 
     & \multicolumn{2}{c|}{\raggedright\parbox[0.14\textwidth]{0.14\textwidth}{Plataformas Compatibles}} 
     & \\ 
     \cline{2-4} \cline{6-12} 
     \multirow{-2}{*}{\rotatebox[origin=cB]{90}{Software}} 
     & \rotatebox[origin=b]{90}{Windows} 
     & \rotatebox[origin=b]{90}{MacOS} 
     & \rotatebox[origin=b]{90}{Linux} 
     & \multirow{-2}{*}{\rotatebox[origin=b]{90}{Licencia}}
    & \rotatebox[origin=b]{90}{C} 
    & \rotatebox[origin=b]{90}{C++} 
    & \rotatebox[origin=b]{90}{C\#} 
    & \rotatebox[origin=b]{90}{Python}
    & \rotatebox[origin=b]{90}{MEL}
    & \rotatebox[origin=b]{90}{Unity} 
    & \rotatebox[origin=b]{90}{Unreal} 
    & \multirow{-2}{*}{\rotatebox[origin=b]{90}{Documentación}} \\
    \hline

    Autodesk Maya 
    & ✓ & ✓ & ✓ 
    &\rotatebox[origin=c]{45}{\$21,041.00}
    & ✗ & ✓ & ✓ & ✓ & ✓
    & ✓ & ✓ & \cite{Maya_Documentation}\\
    \hline
    
    Blender 
    & ✓ & ✓ & ✓ 
    &\rotatebox[origin=c]{45}{Gratuito}
    & ✓ & ✓ & ✗ & ✓ & ✗
    & ✓ & ✓ & \cite{Blender_Documentation}\\
    \hline

    Zbrush 
    & ✓ & ✓ & ✗
    &\rotatebox[origin=c]{45}{\$5,992.82}
    & ✓ & ✓ & ✗ & ✓ & ✗
    & ✓ & ✓ & \cite{Zbrush_Documentation}\\
    \hline
  \end{tabular}
  \caption{Comparativa de los diferentes softwares para el modelado 3D}
  \label{tab:comparativa3D}
\end{table}
\newpage
\subsection{Modelado 2D}

El modelado 2D se refiere a imágenes compuestas por dos dimensiones: ancho y largo, sin profundidad. Se emplean herramientas con entidades geométricas vectoriales como puntos, líneas, arcos y polígonos. En ámbitos como diseño de logos, tipografías e ilustraciones, el diseño bidimensional es esencial. Las animaciones 2D son planas, aunque a veces parecen tener profundidad con el uso de luz y sombra, generalmente limitada al segundo plano.

Tras la definición geométrica de los elementos que componen una escena, aplicamos
propiedades a los objetos que le aportan el color, la rugosidad, brillo etc. (Material).
La textura es una imagen sobrepuesta (mapeada) sobre una forma geométrica. Usada e
para añadir color al elemento, o para modificar las propiedades: La reflectividad, opacidad
(transparencia) o relieve (bump mapping), etc.

\begin{table}[H]
  \centering
  \begin{tabular}{|>{\centering\arraybackslash}m{.1\textwidth}|>{\centering\arraybackslash}m{.03\textwidth}|>{\centering\arraybackslash}m{.03\textwidth}|>{\centering\arraybackslash}m{.03\textwidth}|>{\centering\arraybackslash}m{.09\textwidth}|>{\centering\arraybackslash}m{.02\textwidth}|>
  {\centering\arraybackslash}m{.02\textwidth}|>
  {\centering\arraybackslash}m{.02\textwidth}|>
  {\centering\arraybackslash}m{.02\textwidth}|>
  {\centering\arraybackslash}m{.02\textwidth}|>
  {\centering\arraybackslash}m{.02\textwidth}|>
  {\centering\arraybackslash}m{.02\textwidth}|>
  {\centering\arraybackslash}m{.02\textwidth}|>
  {\centering\arraybackslash}m{.02\textwidth}|>
  {\centering\arraybackslash}m{.03\textwidth}|}
    \hline
     & \multicolumn{3}{c|}{\raggedright\parbox[0.12\textwidth]{0.12\textwidth}{Sistema Operativo}} 
     & 
     & \multicolumn{4}{c|}{\raggedright\parbox[0.15\textwidth]{0.15\textwidth}{Lenguajes de Programación}} 
     & \multicolumn{5}{c|}{\raggedright\parbox[0.12\textwidth]{0.12\textwidth}{Tipos de Archivo}} 
     & \\ 
     \cline{2-4} \cline{6-14} 
     \multirow{-2}{*}{\rotatebox[origin=cB]{90}{Software}} 
     & \rotatebox[origin=b]{90}{Windows} 
     & \rotatebox[origin=b]{90}{MacOS} 
     & \rotatebox[origin=b]{90}{Linux} 
     & \multirow{-2}{*}{\rotatebox[origin=b]{90}{Licencia}}
    & \rotatebox[origin=b]{90}{C++} 
    & \rotatebox[origin=b]{90}{JS} 
    & \rotatebox[origin=b]{90}{Python}
    & \rotatebox[origin=b]{90}{Scheme}
    & \rotatebox[origin=b]{90}{SVG} 
    & \rotatebox[origin=b]{90}{PNG} 
    & \rotatebox[origin=b]{90}{JPEG} 
    & \rotatebox[origin=b]{90}{BMP} 
    & \rotatebox[origin=b]{90}{AI} 
    & \multirow{-2}{*}{\rotatebox[origin=b]{90}{Documentación}} \\
    \hline

    Adobe Ilustrator 
    & ✓ & ✓ & ✗ 
    &\rotatebox[origin=c]{45}{\$3,588.00}
    & ✗ & ✓ & ✗ & ✗
    & ✓ & ✓ & ✓ & ✓ & ✓ 
    & \cite{Ilustrator_User_Guide}\\
    \hline
    
    Gimp 
    & ✓ & ✓ & ✓ 
    &\rotatebox[origin=c]{45}{Gratuito}
    & ✓ & ✗ & ✓ & ✓
    & ✓ & ✓ & ✓ & ✓ & ✗ 
    & \cite{Gimp_Documentation}\\
    \hline

    Inkscape 
    & ✓ & ✓ & ✓ 
    &\rotatebox[origin=c]{45}{Gratuito}
    & ✓ & ✗ & ✓ & ✗
    & ✓ & ✓ & ✓ & ✓ & ✓  
    & \cite{Inkscape_Documentation}\\
    \hline
  \end{tabular}
  \caption{Comparativa de los diferentes softwares para el modelado 2D}
  \label{tab:comparativa2D}
\end{table}
\newpage

