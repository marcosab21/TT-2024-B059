\chapter{Conclusiones}\label{ch:Conclusiones}
El desarrollo del prototipo de simulador de laboratorio de química inorgánica en realidad virtual permitió abordar una necesidad educativa específica mediante una herramienta tecnológica innovadora. A pesar de la existencia de otros simuladores en el ámbito de la química, muchos carecen de una experiencia inmersiva y altamente interactiva. Además, pocos están diseñados específicamente para un público mexicano, limitando su aplicabilidad y efectividad en contextos educativos locales.

El prototipo desarrollado logró implementar cuatro experimentos y un tutorial, permitiendo la realización de prácticas químicas con un nivel aceptable de precisión. Este sistema se centró en representar de manera detallada los aspectos visuales y físicos de las reacciones químicas.
Dichos elementos no solo enriquecen la experiencia del usuario, sino que también contribuyen significativamente a la comprensión de los procesos químicos involucrados.

Una de las características distintivas del simulador es su capacidad para emular experimentos comúnmente realizados en aulas de nivel secundaria. Esta funcionalidad permite a los usuarios interactuar con los experimentos en un entorno virtual, ofreciendo la posibilidad de comparar sus experiencias prácticas en entornos reales. De este modo, el prototipo fomenta un aprendizaje visual, interactivo e inmersivo, que apoya el proceso de enseñanza de conceptos químicos de manera efectiva.

El desarrollo del prototipo presentó retos técnicos y multidisciplinarios, ya que involucró áreas como la química, la enseñanza, el desarrollo de software, el diseño y modelado 3D, y la implementación de tecnologías de realidad virtual. Si bien la creación de los módulos fue compleja, gran parte del esfuerzo se destinó a superar la curva de aprendizaje asociada al enfoque innovador del proyecto.

Finalmente, se concluye que los objetivos establecidos al inicio del proyecto fueron alcanzados. El prototipo cumple con el alcance planteado, al integrar un conjunto de experimentos funcionales, validados por un profesional en química, garantizando así la exactitud de los contenidos. El simulador ofrece una herramienta educativa inmersiva y funcional que puede adaptarse y escalarse en futuras iteraciones para seguir mejorando la enseñanza de la química.