\chapter{Descripción Detallada de los Experimentos}\label{app:Experimentos}

\section{\textit{\textbf{De Básico a Ácido}}} 
\textit{\textsc{Cambio de pH al Añadir dióxido de carbono (\ch{CO2}) a una Solución de hidróxido de sodio (\ch{NaOH}) con Azul de Bromotimol }}

    \textit{\textbf{Introducción  }}
    
    El pH de una solución mide su acidez o basicidad. Un pH bajo indica que la solución es ácida, mientras que un pH alto indica que es básica. El indicador azul de bromotimol cambia de color dependiendo del pH:
    
    \begin{itemize}
        \item Azul en soluciones básicas (pH $>$ 7.6)
        \item Verde en soluciones neutras (pH $\approx$ 7.0)
        \item Amarillo en soluciones ácidas (pH $<$ 6.0)
    \end{itemize} 

    En este experimento, vamos a cambiar el pH de una solución añadiendo primero una base (\ch{NaOH}) para hacerla más básica, y luego introduciremos dióxido de carbono (\ch{CO2}) para que se convierta en ácida.
    
    \textit{\textbf{Objetivo  }}
    
    Observar y comprender el cambio en el pH de una solución básica preparada con hidróxido de sodio (\ch{NaOH}) mediante la adición de dióxido de carbono (\ch{CO2}), utilizando el indicador azul de bromotimol para identificar el cambio de básico a ácido.
    
    \textit{\textbf{Materiales y Reactivos  }}
    \begin{itemize}
        \item Matraz Erlenmeyer (250 ml)
        \item 100 ml de agua destilada
        \item Azul de bromotimol (indicador de pH)
        \item Solución de hidróxido de sodio (\ch{NaOH}) al 0.1 M
        \item Hielo seco (\ch{CO2} sólido) o una fuente de gas \ch{CO2}
        \item Pipeta o gotero
        \item Guantes y gafas de seguridad
    \end{itemize}
    \clearpage
    \textit{\textbf{Procedimiento Experimental  }}
    \begin{enumerate}
        \item \textbf{Preparación de la Solución Básica: }
        \begin{itemize}
            \item Vierte 100 ml de agua destilada en un Matraz Erlenmeyer.
            \item Agrega 2-3 gotas de azul de bromotimol. La solución debería volverse verde (pH neutro).
            \item Añade unas gotas de solución de NaOH (hidróxido de sodio al 0.1 M). Observa cómo el color cambia a azul, indicando que la solución es ahora básica.
        \end{itemize}
    
        \textbf{Ecuación Química  }
        \begin{center}
            \ch{Na2O + H2O -> 2 NaOH}
        \end{center}

        Un hidróxido se forma cuando un óxido metálico reacciona con agua.
        
        \item \textbf{Acidificación de la Solución con \ch{CO2}:}

        Añade un pequeño trozo de hielo seco (\ch{CO2} sólido) o introduce gas \ch{CO2} en la solución. Observa cómo el color de la solución cambia primero a verde y luego a amarillo.

        \textbf{Reacción de \ch{CO2} con Agua:}
        El \ch{CO2} se disuelve en agua formando ácido carbónico, lo que baja el pH:
        
        \begin{center}
            \ch{CO2 + H2O -> H2CO3}
        \end{center}
        
        El ácido carbónico se disocia en iones de hidrógeno (\ch{H+}) y bicarbonto (\ch{HCO3-}), lo que causa la acides de la solución;
        
        \begin{center}
            \ch{H2CO3 -> H+ + HCO3-}
        \end{center}

        Los protones (\ch{H+}) son responsables de la disminución del pH, lo que hace que la solución cambie a un color amarillo, indicando que ahora es ácida.
    \end{enumerate}
    \textit{\textbf{Explicación Química }}  

    En este experimento, el hidróxido de sodio (\ch{NaOH}) se disuelve en agua, creando una solución básica con un pH alto, lo cual se observa como un cambio a color azul en el indicador azul de bromotimol. Lueho, al añadir dióxido de carbono (\ch{CO2}), este se disuelve y reacciona con el agua para formar ácido carbónico (\ch{H2CO3}), el cual se disocia parcialmente, liberando iones \ch{H+} y disminuyendo el pH de la solución.

    Este cambio de pH, visible en el indicador que pasa de azul a amarillo, demuestra cómo el \ch{CO2} puede acidificar una solución básica mediante la formación de ácido.
    
    \textit{\textbf{Resultados }} 
    
    Registrar el tiempo transcurrido hasta el cambio de color en cada experimento. Analizar cómo las variaciones en la concentración y la temperatura afectan este tiempo.  

    \clearpage
    
    \textit{\textbf{Conclusiones}}  
    \begin{itemize}
        \item Se observó que, al agregar \ch{CO2} al agua, la solución pasó de ser básica (azul) a ácida (amarillo).
        \item El cambio de color del indicador azul de bromotimol demostró cómo el \ch{CO2} reduce el pH de la solución.
        \item Este experimento muestra cómo el dióxido de carbono puede acidificar una solución mediante la formación de ácido carbónico.
    \end{itemize}
    
    \textit{\textbf{Medidas de Seguridad }} 
    \begin{itemize}
        \item Usa guantes y gafas de seguridad para evitar el contacto con \ch{NaOH} y hielo seco.
        \item No manipules el hielo seco directamente, usa pinzas o guantes gruesos. 
        \item Trabaja en un área bien ventilada y no inhales el \ch{CO2} directamente.
        \item Desecha los restos de la práctica de acuerdo con las normas de seguridad del laboratorio.
    \end{itemize}

    \textit{\textbf{Cuestionario}} 
    \begin{enumerate}
        \item ¿Qué ocurrió con el color de la solución después de añadir \ch{NaOH}? ¿Por qué?
        \item ¿Por qué se formó ácido carbónico al agregar \ch{CO2} al agua?
        \item ¿Qué indica el cambio de color de azul a amarillo en la solución?
        \item ¿Cómo podría usar el dióxido de carbono para cambiar el pH en otros experimentos?
    \end{enumerate}
    \newpage
\section{\textit{\textbf{La lámpara de magnesio}}} 
\textit{\textsc{Combustión del Magnesio en Hielo Seco}}

    \textit{\textbf{Introducción}}  
    
    El magnesio es un metal altamente reactivo que puede arder incluso en ausencia de oxígeno atmosférico. En este experimento, se observará cómo el magnesio puede reaccionar con el dióxido de carbono sólido (hielo seco), un compuesto comúnmente utilizado como extintor de incendios. Esta reacción demuestra la alta posición del magnesio en la serie de reactividad, desplazando al carbono para formar óxido de magnesio y carbono elemental.  
    
    \textit{\textbf{Objetivos }} 
    \begin{itemize}
        \item Demostrar que el magnesio puede arder en ausencia de oxígeno molecular. 
    
        \item Analizar los productos de la combustión del magnesio en hielo seco.  
        
        \item Comprender la reactividad del magnesio frente al dióxido de carbono.
        
    \end{itemize}
    \textit{\textbf{Materiales y Reactivos }} 
    \begin{itemize}
        \item Bloque de hielo seco (\ch{CO2} sólido). 
    
        \item Soplete o mechero Bunsen.
        
        \item Cincel o herramienta para cavidades. 
        
        \item Pinzas resistentes al calor.
        
        \item Guantes térmicos.
        
        \item Gafas de seguridad.

        \item Pantalla de protección.

        \item Virutas o cinta de magnesio (aproximadamente 5 g).
    \end{itemize}
    \textit{\textbf{Procedimiento Experimental}}  
    \begin{enumerate}
        \item \textbf{Preparación del hielo seco}: Usa un cincel para hacer una cavidad en el bloque de hielo seco de aproximadamente 3 cm de profundidad y 3 cm de ancho. Reserva el material extraído para usarlo como tapa.  
    
        \item \textbf{Colocación del magnesio}: Introduce las virutas de magnesio en la cavidad sin llenarla completamente, dejando al menos 1 cm de espacio por encima del material. 
    
        \item \textbf{Encendido del magnesio}: Utilizando un soplete o mechero Bunsen, calienta el magnesio directamente hasta que se inicie la combustión.
        
        Coloca rápidamente el tapón de hielo seco sobre la cavidad para minimizar la entrada de aire. 
    
        \item \textbf{Observación de la reacción}: Observa cómo el magnesio arde con una luz brillante en el ambiente de dióxido de carbono sólido, produciendo una mezcla de óxido de magnesio (polvo blanco) y carbono elemental (polvo negro).

        \item \textbf{Finalización}: Permite que el residuo se enfríe antes de manipularlo.
    \end{enumerate}
    \textit{\textbf{Explicación Química}}  
    
    La reacción química entre el magnesio y el dióxido de carbono se puede representar con la ecuación:
    
    \begin{center}
        \ch{2 Mg + CO2 -> 2 MgO + C}  
    \end{center}
    
    El magnesio desplaza al carbono del dióxido de carbono debido a su mayor reactividad, formando óxido de magnesio (\ch{MgO}) como producto principal y carbono elemental (\ch{C}) como subproducto. Esto demuestra que el magnesio puede arder en condiciones donde el oxígeno no está disponible. 
    
    \textit{\textbf{Resultados }} 
    
    Durante el experimento, se observó que el magnesio arde intensamente con una luz blanca brillante incluso en ausencia de oxígeno, utilizando el dióxido de carbono sólido como fuente de oxidación. La combustión produjo chispas y partículas expulsadas por la sublimación del hielo seco, dejando como residuo un polvo blanco (óxido de magnesio) y partículas negras (carbono elemental), evidenciando la transformación química esperada. 
    
    \textit{\textbf{Conclusiones }} 
    
    El magnesio demostró ser un metal altamente reactivo capaz de oxidarse utilizando dióxido de carbono como agente oxidante, lo que valida su posición por encima del carbono en la serie de reactividad química. Este experimento refuerza el concepto de que el oxígeno no siempre es imprescindible para una combustión y destaca la capacidad del magnesio para desplazar al carbono, formando óxido de magnesio y carbono elemental como productos finales.
    
    \textit{\textbf{Medidas de Seguridad }} 
    \begin{itemize}
        \item \textbf{Hielo seco}: El hielo seco debe manipularse siempre con guantes térmicos y gafas de seguridad para evitar quemaduras por congelación. Evite el contacto directo con la piel o los ojos, y no almacene el hielo seco en contenedores herméticos o espacios confinados debido al riesgo de acumulación de dióxido de carbono gaseoso que puede causar asfixia o explosión.

        \item \textbf{Magnesio y llama}: El magnesio y su combustión presentan riesgos significativos de quemaduras y proyección de partículas incandescentes. Utilice una pantalla de seguridad, gafas protectoras, guantes resistentes al calor y asegúrese de mantener materiales inflamables alejados del área de trabajo. Evite mirar directamente a la luz intensa producida por la combustión del magnesio para proteger sus ojos.

        \item \textbf{General}: Trabaje siempre en un área ventilada y mantenga al público a una distancia mínima de 2-3 metros detrás de una pantalla de seguridad. Use equipo de protección personal como bata, guantes, gafas y asegúrese de contar con extintores cerca del área de trabajo. Supervise el experimento en todo momento y siga los protocolos de manejo seguro de sustancias químicas y equipos de laboratorio.
    \end{itemize}
    \newpage
\section{\textit{\textbf{Nieve Química}}} 
\textit{\textsc{Síntesis de Cloruro de Amonio a partir de Hidróxido de Amonio y Ácido Clorhídrico}}

    \textit{\textbf{Introducción }} 
    
    El cloruro de amonio (\ch{NH4Cl}) es una sal inorgánica con diversas aplicaciones industriales y científicas. En este experimento, se llevará a cabo su síntesis mediante la reacción de neutralización entre hidróxido de amonio (\ch{NH4OH}) en solución acuosa y ácido clorhídrico (\ch{HCl}) en solución acuosa. La reacción es exotérmica y produce cloruro de amonio sólido, que precipita en forma de cristales.  
    
    \textit{\textbf{Objetivos  }}
    \begin{itemize}
        \item Sintetizar cloruro de amonio a partir de hidróxido de amonio y ácido clorhídrico.  
    
        \item Observar la formación de cristales de cloruro de amonio como producto de la reacción.  
        
        \item Comprender el concepto de reacción de neutralización y su aplicación en la síntesis de sales.  
    \end{itemize}
    \textit{\textbf{Materiales y Reactivos  }}
    \begin{itemize}
        \item Solución de hidróxido de amonio ({\ch{NH4OH}) concentrada (28-30\%)  
    
        \item Solución de ácido clorhídrico (\ch{HCl}) concentrada (37\%)  
        
        \item Vaso de precipitados de 250 ml  
        
        \item Probeta graduada de 50 ml  
        
        \item Varilla de vidrio  
        
        \item Pipeta graduada 
    \end{itemize}
    \textit{\textbf{Procedimiento Experimental }} 
    \begin{enumerate}
        \item Preparación de las soluciones:  medir 32.4 ml de hidróxido de amonio concentrado con una probeta graduada y verterlos en un vaso de precipitados de 250 ml.  
        
        \item Medición del ácido clorhídrico: Medir 25 ml de ácido clorhídrico concentrado con una probeta graduada.  
        
        \item Reacción de neutralización: Lentamente y con precaución, agregar el ácido clorhídrico a la solución de hidróxido de amonio. Agitar suavemente la mezcla. Observar la formación de una densa nube blanca de cloruro de amonio sólido.  
        
        \item Concentración del soluto: Calentar suavemente la solución resultante en la campana extractora hasta concentrar el soluto.  
        
        \item Enfriamiento y cristalización: Verter la solución en una probeta y enfriarla con ayuda de un ventilador. Observar la formación de cristales de cloruro de amonio en forma de estrella en la probeta.
    \end{enumerate}  
    \textit{\textbf{Análisis Químico }}  
    La reacción entre el hidróxido de amonio y el ácido clorhídrico es una reacción de neutralización ácido-base. El ácido clorhídrico (\ch{HCl}) dona un protón (\ch{H+}) al hidróxido de amonio (\ch{NH4OH}), que actúa como base aceptando el protón. Esto resulta en la formación de agua (\ch{H2O}) y cloruro de amonio (\ch{NH4Cl}).   
    
     El cloruro de amonio formado es una sal soluble en agua, pero a medida que la solución se concentra por evaporación, se alcanza el punto de saturación y el cloruro de amonio comienza a precipitar en forma de cristales. El enfriamiento rápido de la solución favorece la formación de cristales más pequeños y numerosos, como los cristales en forma de estrella observados en el experimento.   
    
    \textit{\textbf{Resultados Esperados  }} 
    
    \textbf{Reacción exotérmica:} Al mezclar las soluciones de hidróxido de amonio y ácido clorhídrico, se espera observar un aumento de temperatura en el vaso de precipitados debido a la liberación de calor.   
    
    \textbf{Formación de humo blanco:} Al inicio de la reacción, se espera observar la formación de una densa nube blanca en el vaso de precipitados. Este humo blanco está compuesto por pequeñas partículas de cloruro de amonio sólido que se forman rápidamente en la reacción.   
    
    \textbf{Precipitación de cristales:} Al concentrar el soluto por evaporación y enfriar la solución, se espera observar la formación de cristales de cloruro de amonio en forma de estrella en la probeta.   
    
    \textit{\textbf{Conclusiones}}   
    
    Este experimento permite demostrar la reacción de neutralización entre un ácido y una base, así como la formación de una sal (cloruro de amonio) como producto de la reacción. Además, ilustra la importancia de la concentración y la temperatura en la solubilidad y la cristalización de las sales. La formación de cristales de cloruro de amonio en forma de estrella es un ejemplo interesante de cómo las condiciones experimentales pueden influir en la morfología de los cristales.
    \newpage
\section{\textit{\textbf{La Bruja de Bromo}}}
\textit{\textsc{Reacción de Oxidación-Reducción entre Bromo y Aluminio}}  

    \textit{\textbf{Introducción }} 
    
    El presente experimento tiene como objetivo ilustrar la naturaleza altamente exotérmica de la reacción redox entre bromo (\ch{Br2}) y aluminio (\ch{Al}), la cual resulta en la formación de bromuro de aluminio (\ch{AlBr3}). Este proceso químico es un ejemplo ilustrativo de la reactividad de los halógenos y la susceptibilidad de los metales a la oxidación.  
    
    \textit{\textbf{Objetivos }} 
    \begin{itemize}
        \item Demostrar la reacción vigorosa que ocurre entre el bromo líquido y el aluminio metálico.  
    
        \item Observar la formación de bromuro de aluminio como producto de la reacción.  
        
        \item Analizar los principios de oxidación y reducción en el contexto de esta reacción específica. 
    \end{itemize}
    \textit{\textbf{Materiales y Reactivos}}  
    \begin{itemize}
        \item Bromo (\ch{Br2}) líquido (¡Precaución! El bromo es altamente corrosivo y tóxico)  
    
        \item Lámina de aluminio  
        
        \item Tubo de ensayo  
        
        \item Pipeta graduada  
        
        \item Pinzas de laboratorio  
        
        \item Vidrio de reloj  
        
        \item Campana extractora de gases  
        
        \item Guantes de nitrilo  
        
        \item Gafas de seguridad  
    \end{itemize}
    \textit{\textbf{Procedimiento Experimental }} 
    \begin{enumerate}
        \item Preparación: En el entorno controlado de una campana extractora de gases, se depositará una pequeña cantidad de bromo líquido (unas pocas gotas) en un tubo de ensayo.  
        
        \item Adición de aluminio: Utilizando pinzas de laboratorio, se introducirá con precaución un pequeño fragmento de lámina de aluminio en el tubo de ensayo que contiene el bromo.  
        
        \item Observación: Se procederá a observar detenidamente la reacción inmediata y vigorosa que se desencadena. Se espera una reacción exotérmica, acompañada de liberación de energía en forma de luz y calor. Se anticipa la formación de humo blanco correspondiente al bromuro de aluminio (\ch{AlBr3}). 
    \end{enumerate}
    \newpage
    \textit{\textbf{Análisis Químico}}  
    
    La reacción química que tiene lugar entre el bromo y el aluminio se puede representar mediante la siguiente ecuación estequiométrica:  
    
    \ch{2 Al_{(s)} + 3 Br2_{(aq)} -> 2 AlBr3_{(s)}}  
    
    En este proceso, el aluminio experimenta oxidación, perdiendo electrones, mientras que el bromo se reduce, ganando electrones. El resultado neto es la formación de bromuro de aluminio, un sólido de color blanco.  
    
    \textit{\textbf{Resultados Esperados }} 
    
    Se prevé observar una reacción rápida y enérgica, caracterizada por la emisión de luz y calor. La formación de humo blanco de bromuro de aluminio será un indicador visual de la ocurrencia de la reacción.  
    
    \textit{\textbf{Conclusiones}}  
    
    Los resultados obtenidos permitirán corroborar la naturaleza altamente reactiva del bromo y la propensión del aluminio a la oxidación. La reacción servirá como modelo para ilustrar los principios fundamentales de las reacciones redox y la transferencia electrónica entre especies químicas.  
    
    \textit{\textbf{Medidas de Seguridad }} 
    \begin{itemize}
        \item ¡Precaución Extrema! El bromo es una sustancia altamente corrosiva y tóxica. Su manipulación debe realizarse exclusivamente en una campana extractora de gases y utilizando equipo de protección personal adecuado, incluyendo guantes de nitrilo y gafas de seguridad.  
        
        \item Es imperativo evitar el contacto directo con el bromo. En caso de contacto accidental con la piel, se debe lavar inmediatamente la zona afectada con abundante agua y buscar atención médica si es necesario.  
        
        \item La inhalación de los vapores de bromo debe ser evitada en todo momento.  
        
        \item El experimento debe ser realizado bajo la supervisión de personal debidamente capacitado y experimentado en el manejo de sustancias químicas peligrosas.  
        
        \item Es fundamental contar con un extintor de incendios adecuado y estar familiarizado con los protocolos de emergencia en caso de cualquier eventualidad. 
    \end{itemize}