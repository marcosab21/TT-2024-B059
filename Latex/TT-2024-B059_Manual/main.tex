\documentclass{ipn}

\usepackage{ipnstyle}
\usepackage{lipsum}
\usepackage{graphicx}
\usepackage{float}
\usepackage{caption}
\usepackage{subcaption}
\usepackage{bookmark}
\usepackage{hyperref}
\usepackage{minted}
\usepackage{times}
\usepackage{multicol}
\usepackage{multirow}
\usepackage{chemformula}
\usepackage{float}
\usepackage[utf8]{inputenc}
\usepackage{pifont}
\usepackage{newunicodechar}
\usepackage{array}
\usepackage{longtable}
\usepackage{wrapfig}
\usepackage{ragged2e}
\usepackage{pdfpages}
\newunicodechar{✓}{\ding{52}}
\newunicodechar{✗}{\ding{56}}
% \usepackage{cite}

\author{García Aguayo Marcos Martí Sandino Mictlantecuhtli}
\title{Manual De Usuario \\ Simulador de Laboratorio de Química Inorgánica en Realidad Virtual}
\schoolname{Escuela Superior de Cómputo}
 


\hypersetup{
    colorlinks=false,
    pdfauthor={García Aguayo Marcos Martí Sandino Mictlantecuhtli},
    pdftitle={Prototipo de Simulador de Laboratorio de Química Inorgánica en Realidad Virtual},
    pdfsubject={Thesis},
    pdfkeywords={Química Inorgánica, Realidad Virtual, Simulador, Software Interactivo}
    pdfproducer={Latex with hyperref},
    pdfcreator={pdflatex}
}

\pagestyle{headings}
\begin{document}
\frontmatter
    \maketitle
    \tableofcontents
    \listoffigures
\mainmatter
    % Aplicar el estilo de página a todo el documento
\pagestyle{plain}
\centering
\textbf{\textit{\large{Prototipo de Simulador de Laboratorio \\ de Química Inorgánica en Realidad Virtual}}}\\

García Aguayo Marcos Martí Sandino Mictlantecuhtli,\\ M. en E. Saul De La O Torres, Dr. Gabriel Sepúlveda Cervantes\\

Escuela Superior de Cómputo I.P.N. México CDMX.\\
Tel. 57-29-6000 ext 52000 y 52021. E-mail: \href{mailto:mgarciaa1711@alumno.ipn.mx}{mgarciaa1711@alumno.ipn.mx}\newline

\begin{multicols}{2}
[
    \section*{hola}
]
\end{multicols}
\end{document}